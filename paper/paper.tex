% This project is part of the Continuous-Function Estimator project
% Copyright 2019 the authors.

% to-do
% -----
% - get yourself an ORCID and use it.
% - Is ``vectorized'' the right word? It's what I say, but now I'm doubting it. We could go with linear-regression. Or affine-invariant. Or unbinned. Or continuous-function?

% style notes
% -----------
% - line break at sentence breaks? Or more frequently? So git diff works good?

\documentclass[twocolumn]{aastex62}

\usepackage[sort&compress]{natbib}
\usepackage{xspace}

% aastex parameters
%%\hypersetup{linkcolor=red,citecolor=green,filecolor=cyan,urlcolor=magenta}
\received{XXX}
%\revised{not yet}
\accepted{YYY}
%\submitjournal{ApJ}
\shorttitle{A Continuous Correlation Function Estimator}
\shortauthors{Storey-Fisher and Hogg}

% typography
\newcommand{\cf}{2pcf\xspace} %2pF? 2PCF? %TODO: fix spacing after
\newcommand{\Est}{The Continuous-Function Estimator\xspace}
\newcommand{\est}{the Continuous-Function Estimator\xspace}

% affiliations
\newcommand{\ccpp}{\affiliation{%
    Center for Cosmology and Particle Physics,
    Department of Physics,
    New York University}}
\newcommand{\flatiron}{\affiliation{%
    Flatiron Institute, Simons Foundation}}
\newcommand{\cds}{\affiliation{%
    Center for Data Science,
    New York University}}
\newcommand{\mpia}{\affiliation{%
    Max-Planck-Institut f\"{u}r Astronomie, Heidelberg}}


\begin{document}\sloppy\sloppypar\raggedbottom\frenchspacing

\title{\textbf{A Continuous-Function Estimator of the Correlation Function for Large-Scale Structure}}
%\title{\textbf{A Continuous Correlation Function Estimator for Large-Scale Structure}}
%\title{\textbf{A Continuous-Function Correlation Function Estimator for Large-Scale Structure}}
%\title{\textbf{Projecting the Correlation Function onto Continuous Functions for Large-Scale Structure Surveys}
%\title{\textbf{A Continuous Representation of the Correlation Function Estimator for Large-Scale Structure}}
%\title{\textbf{Correlation Function Estimation with Continuous Functions for Large-Scale Structure}}
%\title{\textbf{Estimating the Two-Point Correlation Function with Continuous Functions}}
%\title{\textbf{Projecting Two-Point Correlations onto General Basis Functions: A New Estimator For Large-Scale Structure}}
%\title{\textbf{A Correlation Function Estimator with General Basis Functions for Large-Scale Structure}}
%\title{\textbf{A Generalized Correlation Function Estimator For Large-Scale Structure}}
%\title{\textbf{No More Bins:\\ A Vectorized Correlation Function Estimator For Large-Scale Structure}}
%\title{A Generalized Correlation Function Estimator for Galaxy Surveys}

%Generalized Estimator?
%Projected Estimator?
%Continuous Estimator?
%Continuous-Function Estimator?
%Linear-Regression Estimator?
%Vectorized estimator?

\author[0000-0001-8764-7103]{Kate Storey-Fisher}
\ccpp

\author[0000-0003-2866-9403]{David W. Hogg}
\ccpp
\cds
\mpia
\flatiron

\begin{abstract}\noindent
% Context
The two-point correlation function (\cf) is the most important statistic in structure formation.
Current estimators of the \cf, used to measure the clustering of density field tracers, have significant limitations.
The standard Landy-Szalay (LS) estimator evaluates the \cf in bins of radial separation between objects, and the choice of bins introduces a trade-off between bias and variance.
% Aims
We present a new, generalized estimator, \est, for the \cf that obviates binning in separation or any other property.
% Results
\Est replaces the binned pair counts of LS with a continuous-function representation in the form of a linear superposition of basis functions.
These functions can take advantage of the known form of the \cf, and can depend on other properties of the pairs in addition to their separation.
Inspired by linear least-squares fitting, \est marks the numerator terms of LS with vectors of these functions, and the denominator with a tensor of their outer products.
The estimator outputs the best-fit linear combination of basis functions to describe the \cf, in the same limit in which LS is justifiably optimal for binned applications. 
% Results
We show that it can estimate the clustering of artificial data in representations that provide more accuracy with fewer basis functions than LS, thus reducing requirements on mocks for covariance estimation.
We apply \est to the Luminous Red Galaxy sample of the Sloan Digital Sky Survey (SDSS), and show can extract a compact set of Fisher-optimal statistics for measuring the location of the baryon acoustic feature.
We discuss other applications and limitations of \est for present and future studies of large-scale structure, including determining the dependence of clustering on galaxy properties and potentially unifying real-space and Fourier-space approaches by using a Fourier basis representation.
\end{abstract}

\keywords{cosmology: large-scale structure --- galaxies: statistics} %?

\section{Introduction}

The large-scale structure (LSS) of the universe is a critical probe of fundamental cosmology. 
It encodes information about the physics of the early universe and the subsequent expansion history.
In particular, the LSS provides a probe of the Baryon Acoustic Oscillations (BAO), density fluctuations resulting from baryon-photon coupling in the early universe.
The distance traveled by these waves imprints a feature on the statistical description of the LSS, which can be used to determine the characteristic BAO length scale \citep{EisensteinHu1998}.
The LSS also contains the signature of redshift-space distortions caused by the peculiar velocities of galaxies, which probe the growth rate of structure \citep{Kaiser1987}.
Additionally, the LSS can be used to constrain galaxy formation in conjunction with models of galaxy bias (e.g. \citealt{Hamilton1988}). %?
With current observations, the LSS is well-described by a cold dark matter model with a cosmological constant, the standard $\Lambda$CDM model.
Upcoming galaxy surveys will observe larger volumes with improved measurements, allowing us to test $\Lambda$CDM to even higher precision.

We characterize the LSS by using luminous sources to trace the underlying matter density field.
These tracers are often to taken to be galaxies, but can also be galaxy clusters, quasars and other sources.
% should i say that hereafter we will take them to be galaxies?
The clustering of these objects is measured with two-point statistics, namely the power spectrum $P(k)$ and the two-point correlation function (\cf). %name xi_r?
These characterize the clustering in Fourier space and real space, respectively, with the \cf defined as the Fourier Transform of the power spectrum; in principle, they contain the same information. %show eqn? 
However, in practical applications the survey boundaries introduce nontrivial issues in computing these statistics, leading to diverging approaches to their computation with a significant difference in expense.
The \cf requires more computational power and extra survey products, but it is an incredibly useful tool; for instance, it well-suited to the analysis of the BAO feature which manifests at a single scale in real space.

The \cf measures the excess probability that any two galaxies are separated by a given distance, compared to a uniform distribution; effectively, it characterizes the strength of clustering at a given spatial scale. 
% need to discuss statistical homogeneity & isotropy? (ergodicity?)
In calculating the \cf, the boundaries of the surveys prevent us from directly summing pair counts due to nontrivial edge effects.
To account for the survey boundaries as well as corrupted regions due to issues such as bright foreground stars, a set of random points are Poisson-distributed within the acceptable survey window. 
The pairwise correlations of these points, which by construction are unclustered, are used to normalize out the survey window when estimating the \cf of the clustered data.
Typically, this requires random points on the order of 10-100 times the number of data points, making the random correlations the limiting factor in \cf computation.

The \cf is computed in bins of radial separation, meaning that in practice it measures the volume average of the \cf over the bin.
The choice of bins requires a trade-off between bias and variance: fewer bins may bias the result, while more bins increases the variance of measurement.
Finite-width bins also result in a loss of information about the property in which one is binning. 
Especially as we strive to extreme precision in large-scale structure analyses, we should be maximizing the information we extract from the data.
Moreover, we note that standard estimators are fundamentally biased due to this finite binning; we will address this in a future work. %TODO: figure out this wording

%some papers have equations (even this eqn) in intro - but should bump it to motivation and just explain?
The current standard estimator was proposed by \cite{LandySzalay1993}, hereafter LS, which is based on summing all data pairs $DD$ with a given separation and using data-random pairs $DR$ and random pairs $RR$ to correct for the survey boundary. The correlation function $\xi_k(r)$ for the $k^\mathrm{th}$ separation bin is
%assume we should keep this notation even though it's clunky, as it's recognizable
%should we use estimator notation with a hat on xi?
\begin{equation}
\xi_k(r) = \frac{DD_k(r) - 2DR_k(r) + RR_k(r)}{RR_k(r)}.
\end{equation}
Compared with other estimators based on combinations of $DD$, $DR$ and $RR$, LS has been shown to have the lowest bias and variance \citep{Kerscher2000}.
%is this necessary? not sure where it fits
Estimators of the \cf must also take into account the imperfect nature of the survey area that is in the window, including the target completeness and fiber collisions; typically each galaxy pair is assigned a weight based on these.

Variations on the random catalog pair count method have been proposed in recent years.
\cite{Demina2016} replaced the $DR$ and $RR$ terms with an integral over the probability map, reducing computation time and increasing precision.
An estimator proposed by \cite{VargasMagana2013} iterates over sets of mock catalogs to find an optimal linear combination of data and random pair counts, reducing the bias and variance.
The marked correlation function \citep{WhitePadmanabhan2009} sums weights based on properties of the galaxies one is interested in, such as the local density or galaxy luminosity, and avoids the use of a random catalog.
The estimators described so far have all taken probabilistic approaches; some have also taken a likelihood approach.
\cite{BaxterRozo2013} introduced a maximum likelihood estimator for the \cf, which achieves lower variance compared to the LS estimator, enabling finer binning and requiring a smaller random catalog for the same precision.

These estimators present improvements to LS, but they all still require binning in separation.
Some require additional computational costs or layers of complexity, so the standard formulation of LS continues to be the default estimator used in analyses.

In this paper, we present a new estimator for the correlation function, \est, which generalizes the LS estimator to produce a continuous estimation of the \cf. 
\Est projects the galaxy pairs onto a set of continuous basis functions and computes the best-fit linear combination of these functions.
The basis representation can depend on the pair separation as well as other desired properties, and can also utilize the known form of the \cf.
For top-hat basis functions, \est exactly reduces to the LS estimator. 
This estimator removes the need for binning and allows for the \cf to be represented by fewer basis functions, requiring fewer mock catalogs to compute the covariance matrix.
It is particularly well-suited to the analysis of LSS features such as the BAO peak; we find that we can more accurately locate the peak with fewer components.

%spell out acronyms again that were defined in the abstract?
This paper is organized as follows. In Section \ref{sec:motiv}, we motivate and derive our estimator. In Section \ref{sec:est}, we prove its correctness and describe our implementation, and demonstrate its application on a simulated data set. Our results on the SDSS LRG sample are shown in Section \ref{sec:app}. We discuss the implications and other possible applications in Section \ref{sec:discuss}. 

\section{Motivation} \label{sec:motiv}

The Landy-Szalay estimator is justified by\ldots

Estimating clustering is closely related to least-squares fitting.

From these, we can infer the form of the estimator.

% TODO: fix this capitalization issue
\section{\Est} \label{sec:est}

We derive/define this new estimator.

We show that it reduces to the Landy-Szalay estimator.

We prove that the estimator is invariant under affine transformations.

We show that in the limit of infinitesimal bins\ldots

These show that the estimator is correct.

We show that it extracts Fisher-optimal information.

We implement this estimator based on the correlation function package \texttt{corrfunc} by \citep{citeXXX} \ldots
The code is open-source and available at XXX.

We show its performance on a simulated Gaussian random field.

\section{Application to survey data} \label{sec:app}

We apply the estimator to the SDSS DR3 sample from \cite{Eisenstein2005}.

The reconstructed dataset we use is described in \cite{Kazin2010}.

We use \est to linearize around the BAO feature. 

\section{Discussion} \label{sec:discuss}

We estimated the \cf with fewer components to achieve the same level of precision.

We note that our estimator doesn't rely on LS; it could be implemented for various linear combination of pair counts, given certain properties. %aka the tensor term

Our performance is by definition the same (marginally more) compared to traditional estimators due to the limiting factor of pair-finding.

We discuss the impact on computing the covariance matrix.

We discuss the relation of our estimator to the optimized estimator of \cite{VargasMagana2013}, which also uses a linear combination approach.

In the future, we could simultaneously estimate the correlation function and the power spectrum.

We could also investigate: gradients, anomalies, growth of structure, directly estimating the cosmological parameters, calibrating for systematics, clustering dependence galaxy properties


\acknowledgements
It is a pleasure to thank\ldots

%need style file error
%\bibliographystyle{apj} 
\bibliography{paper}

\end{document}
